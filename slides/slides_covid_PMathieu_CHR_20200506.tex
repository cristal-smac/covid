\documentclass[a4paper]{cours-bdd}

\cristal

\title{COVID Simulation}

\begin{document}

\frame{\titlepage}


% --------------------------------------


\begin{frame}
  \hfill \
  \begin{center}
    \Huge
    Préambule
  \end{center}
  \hfill \

\end{frame}



% --------------------------------------

\begin{frame}[fragile]
  \frametitle{Modéliser les épidémies : pour faire quoi ?}

  Aider à comprendre.\\
  Aide à la décision.\\
  Essayer de répondre aux grandes questions. \\
  \begin{itemize}
  \item Combien de temps va durer l’épidémie ?
  \item Combien de personnes seront infectées au cours de la crise ? 
  \item Combien de personnes décèderont au cours de la crise ? 
  \item Combien de personnes doivent être immunisées ?
  \item Quand mettre en place un ou des confinements ?
  \item Quelle durée doivent avoir les confinements ?
  \item Quand arrivera t-on à saturation des hopitaux ?
  \end{itemize}
  
\end{frame}

  
% --------------------------------------

\begin{frame}[fragile]
  \frametitle{Différentes approches}

  %\fbox{}
  \begin{minipage}[b]{0.45\columnwidth}
    \vspace{0.0cm}
    \includegraphics[width=\linewidth]{approcheMath.png} \\
    Approche Mathématique 
      \end{minipage}    
  \hfill \ 
  \begin{minipage}[b]{0.45\columnwidth}
        \vspace{0.0cm}
        \includegraphics[width=0.8\linewidth]{approcheNetwork.png} \\
        Approche par réseaux sociaux
      \end{minipage}
      \ 

  \begin{minipage}[t]{0.45\columnwidth}
        \vspace{0.0cm}
        \includegraphics[width=\linewidth]{modeleSIR.png} \\
        Approche par flux
  \end{minipage}
  \hfill \
  \begin{minipage}[t]{0.45\columnwidth}
        \vspace{0.0cm}
        \includegraphics[width=0.6\linewidth]{approcheSMA.png} \\
        Approche centrée individus
  \end{minipage}
  \
  
\end{frame}

% --------------------------------------


\begin{frame}
  \hfill \
  \begin{center}
    \Huge
    Notre situation actuelle
  \end{center}
  \hfill \

\end{frame}



% --------------------------------------

\begin{frame}[fragile]
  \frametitle{Notre modèle - notre méthodologie}
  \begin{itemize}
  \item Approche par flux (compartiments) ``classique'' \\ (tout comme Inserm ou Pasteur)
    \includegraphics[width=0.6\linewidth]{modeleSIR.png} \\
  \item Prise en compte Asymptomatiques, Graves, Décès, etc ...
    
    \bigskip    
    \textbf{Points originaux:}
  % \item SIGRM avec contagiosité variable tout au long du temps
  \item Contagiosité variable tout au long du temps
  % \item SIAGRM (avec asymptomatiques et non asymptomatiques)
  \item Techniques d'IA pour identifier les bonnes valeurs des paramètres 
  \end{itemize}

  \begin{block}{}
    \begin{center}
  Dans notre approche, c'est l'IA qui ajuste les paramètres !
\end{center}
\end{block}

  
\end{frame}



% --------------------------------------

\begin{frame}[fragile]
  \frametitle{La validation}
  \begin{itemize}
  \item Un modèle s'appuie sur des paramètres
  \item Plus il y a de paramètres plus c'est facile de coller aux données (overfitting)
  \end{itemize}

  \bigskip
  
  \textbf{Question : Comment valider le modèle}

  \begin{itemize}
  \item Par autorité
    
  \item Par calibration
    \begin{itemize}
    \item Par les faits stylisés propres à une épidémie 
      (croissance exponentielle, puis décroissance)
    \item Par sa capacité à reproduire les situations passées % (est-ce qu'on peut régler les paramètres pour que le modèle montre la situation actuelle)
  \end{itemize}

  \end{itemize}
\end{frame}


% --------------------------------------

\begin{frame}[fragile]
  \frametitle{Faits stylisés}
  
  \begin{center}
            \includegraphics[width=1.0\linewidth]{fig3_sirgm.png} \\
  \end{center}

  
\end{frame}


% --------------------------------------

\begin{frame}[fragile]
\frametitle{Coller à la réalité : Réa}

  \begin{center}
    \includegraphics[width=1.0\linewidth]{figure1.jpg} \\
    {\tiny à partir des données ministère santé \texttt{data.gouv.fr} \\
      {\tiny Selon notre modèle R0=2.0 avant confinement, R0=2.7 durant la premiere semaine, R0=0.7 durant le confinement}}

  \end{center}
  
\end{frame}


% --------------------------------------

\begin{frame}[fragile]
\frametitle{Coller à la réalité : Décès}

  \begin{center}
    \includegraphics[width=1.0\linewidth]{figure2.jpg} \\
    {\tiny à partir des données ministère santé \texttt{data.gouv.fr}} \\
    {\tiny Selon nous R0=2.0 avant confinement, R0=2.7 durant la premiere semaine, R0=0.7 durant le confinement}

\end{center}
  
\end{frame}


% --------------------------------------

\begin{frame}[fragile]
\frametitle{Projection sur Réa : 3 hypothèses}

  \begin{center}
    \includegraphics[width=1.0\linewidth]{figure3.jpg} \\
    {\tiny à partir des données du ministère de la santé \texttt{data.gouv.fr}}
  \end{center}
  
\end{frame}

% --------------------------------------

\begin{frame}[fragile]
\frametitle{Projection sur Décès : 3 hypothèses}

  \begin{center}
    \includegraphics[width=1.0\linewidth]{figure4.jpg} \\
    {\tiny à partir des données du ministère de la santé \texttt{data.gouv.fr}}
  \end{center}
  
\end{frame}

% --------------------------------------

\begin{frame}[fragile]
\frametitle{Propagation intermédiaire : rester < 7000 lits ?}

  \begin{center}
    \includegraphics[width=1.0\linewidth]{figure5.jpg} \\

        \textbf{Confinement necessaire au 20 juin}
  \end{center}
  
\end{frame}

% --------------------------------------

\begin{frame}[fragile]
\frametitle{Propagation pessimiste : comment rester < 7000 lits ?}
\begin{center}
\includegraphics[width=1.0\linewidth]{figure6.jpg} \\

\textbf{Confinement nécessaire au 27 mai}

  \end{center}
  
\end{frame}

% --------------------------------------

\begin{frame}[fragile]
\frametitle{Et si on décale d'1 jour ?}

\begin{center}
    \includegraphics[width=1.0\linewidth]{figure7.jpg} \\

      \textbf{On passe à 8000 Réa !!}
  \end{center}
  
\end{frame}

% --------------------------------------

\begin{frame}[fragile]
\frametitle{Et si le 13 mai on fait légèrement la fête ?}

  \begin{center}
    \includegraphics[width=1.0\linewidth]{figure8.jpg} \\
    {\tiny à partir des données ministère santé \texttt{data.gouv.fr}}
  \end{center}
  
\end{frame}



% --------------------------------------

\begin{frame}[fragile]
\frametitle{Comment s'en sortir sans re-confinement ?}

  \begin{center}
    \includegraphics[width=1.0\linewidth]{figure9.jpg} \\

    \textbf{Seule l'horizontale (R0=1) garantit un non re-confinement  !!\\
    {\tiny Selon nous R0=2 avant confinement, R0=2.7 durant la premiere semaine, R0=0.7 durant le confinement}}
  \end{center}
  
\end{frame}



% --------------------------------------

\begin{frame}[fragile]
\frametitle{Comment s'en sortir sans re-confinement ?}

  \begin{center}
    \includegraphics[width=0.8\linewidth]{challenges_28_04_Allemagne.png} \\
\href{https://www.challenges.fr/france/en-direct-deconfinement-six-francais-sur-dix-ne-font-pas-confiance-au-gouvernement_707222}{Challenges 28-04-2020}
  \end{center}
  
\end{frame}




% --------------------------------------

\begin{frame}[fragile]
  \frametitle{Comparaison avec Pasteur}

  \begin{center}
    \includegraphics[width=\linewidth]{Inserm+Pasteur_HdF.png}

    {\scriptsize Note 30 du Groupe de modélisation de l’épidémie COVID, 25 avril 2020}

    \bigskip
    
    \begin{block}{}
      \begin{center}
        Différence : chez nous c'est l'IA qui trouve les R0 !
      \end{center}
    \end{block}
  \end{center}

\end{frame}



% --------------------------------------


\begin{frame}
  \hfill \
  \begin{center}
    \Huge
    Conclusion
  \end{center}
  \hfill \

\end{frame}

% --------------------------------------

% \begin{frame}[fragile]
%   \frametitle{Mise en garde}

%   Ne jamais perdre de vue que  :
%   \begin{itemize}
%   \item Un modèle n'est qu'une abstraction de la réalité
%   \item L'important n'est pas d'avoir le plus de paramètres, mais de trouver les plus pertinents (l'essence du problème étudié)
%   \item Pour que les thématiciens s'approprient un modèle, il faut qu'il soit simple (exemple le modèle SIR : 3 boites citées dans des centaines de travaux en épidémio)
%   \end{itemize}

%   \bigskip
  
%   \begin{block}{}
%     \begin{center}
%   Torturez un modèle, il finit toujours par avouer !
% \end{center}
% \end{block}

% \end{frame}

% % --------------------------------------

\begin{frame}[fragile]
  \frametitle{Nous sommes complémentaires !}

  \begin{itemize}
  \item Notre modèle est adaptable en tout point
  \item Il y a surement plein de biais .... \\ Nous avons besoin de
    collaborer
  \end{itemize}

  \bigskip

  
  \begin{itemize}
    \item Nous sommes juste des modélisateurs
    \item Nous avons besoin de votre experience de spécialistes
    \item Nous avons besoin de données régionales/départementales/CHR avant le 18 mars (ARS ?)
    \item Qu'attendez vous de nous ?
    \end{itemize}

  
\end{frame}


\end{document}

